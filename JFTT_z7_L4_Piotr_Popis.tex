\documentclass[11pt]{article}
\usepackage[T1]{fontenc}
\usepackage[left=12mm,right=12mm,top=0.5in,bottom=0.5in]{geometry}
\usepackage{mathtools}
\usepackage{cancel}
	
\begin{document}
\title{Zadanie 7 Lista 4}
\author{Piotr Popis, 245162}
\maketitle
\centering

\begin{flushleft}
\section{Treść}
Czy język tych słów nad alfabetem \{1,2,3,4\}, które mają tyle samo symboli 1 co 2 i tyle samo symboli 3 co 4 jest bezkontekstowy?
\end{flushleft}

\begin{flushleft}
\section{Rozwiązanie}
Załóżmy, że język L jest bezkontekstowy.
Wtedy skorzystamy z lematu o pompowaniu:\\
Istnieje stała $n$ taka, że jeśli $z$ $\epsilon$ $ L$ $\wedge$ $| z |$ $\geq$ $n$  oraz \\
\begin{center}
$z = uwvxy$, gdzie $|vw|$ $\geq$ $1$  $\wedge$ $|vwx|$ $\leq$ $n$
\end{center}
to wtedy 
\begin{center}
$z = uw^{i}vx^{i}y$ $\epsilon$ $L$, dla każdego $i$ $\epsilon$ $N\cup\{0\}$ 
\end{center}
Niech $n$ będzie dowolną liczbą naturalną.\\
Rozważmy słowo $z$ = $1^{n}3^{n}2^{n}4^{n}$ oczywiście $z$ $\epsilon$ $L$, bo  $| z |_{1}$=$| z |_{2}$=$| z |_{3}$=$| z |_{4}$=$n$ oraz $| z |$= $4n$ $\geq$ $n$ teraz przyjmijmy, że $|v|$ = $k-1$ i rozważmy możliwe postaci $vwx$\\
\subsection{\quad $vwx$ = $1^{+}$} 
Dla i = 0 mamy sprzeczność, bo $|z_1|_{1}=n-k-|x| \neq n = |z_1|_{2}$, analogicznie dla pozostałych przypadków postaci $m^+$, gdzie m $\epsilon$ $\{2,3,4\}$

\subsection{\quad $vwx$ = $1^{+}3^{+}$} 
Teraz jeśli $v$ = $1^+$ to dla i = 0 $z_2$ \( \cancel{\epsilon}\) L, bo $|z_1|_{1}=n-k-|x|_1 \neq n = |z_1|_{2}$\\
,a jeśli $v$ = $1^{+}3^{+}$ to dla i = 0 $z_3$ \( \cancel{\epsilon}\) L, bo$|z_1|_{1}=n-k-|x|_1 \neq n = |z_3|_{2}$\\
Analogicznie dla pozostałych przypadków $vwx$ = $3^+2^+$ oraz $vwx$ = $2^+4^+$\\
\section{Wniosek}
\begin{center}
Doprowadziliśmy do sprzeczności zatem L nie jest językiem bezkontekstowym.
\end{center}

\end{flushleft}

\end{document}